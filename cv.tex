%%%%%%%%%%%%%%%%%%%%%%%%%%%%%%%%%%%
% My Own template for my Academic Curriculum Vitae: a short and clean version easily readable.
%
% Authored By André Luis Albuquerque dos Reis
%%%%%%%%%%%%%%%%%%%%%%%%%%%%%%%%%%%%

\documentclass[10pt,a4paper]{article}

%---------------------
% Loading Packages
%---------------------

%--- non-ASCII characters
\usepackage[utf8]{inputenc}
\usepackage[english,portuguese]{babel}
\usepackage[TU]{fontenc}
%--- Icons 
\usepackage{fontawesome5}
\usepackage{academicons}
%--- For fancy and multipage tables
\usepackage{tabularx}
%--- For new environments
\usepackage{environ}
%--- Manage dates and times
\usepackage{datetime}
%--- Set the page margins
\usepackage{geometry}
%--- To get the total page numbers
\usepackage{lastpage}
%--- Control spacing in enumerates
\usepackage{enumitem}
%--- Use custom colors
\usepackage[usenames,dvipsnames]{xcolor}
%--- Configure section titles
\usepackage{titlesec}
%--- Fancy header configuration
\usepackage{fancyhdr}
%--- Control PDF metadata and links
\usepackage{hyperref}
%--- Disable hyphenation
\usepackage[none]{hyphenat}

%%%%% End loading packages

%------------------------------------
% Begin defining some new commmands
%------------------------------------

%--- Defining simple commands for institutions
\newcommand{\UERJ}{Universidade do Estado do Rio de Janeiro (UERJ)}
\newcommand{\CBPF}{Centro Brasileiro de Pesquisas F\'{i}sicas (CBPF)}
\newcommand{\ON}{Observat\'{o}rio Nacional (ON)}
\newcommand{\UENF}{Universidade Estadual do Norte Fluminense (UENF)}
\newcommand{\DGAP}{Departamento de Geologia Aplicada}
\newcommand{\FGEL}{Faculdade de Geologia}
\newcommand{\CNPq}{Centro Nacional de Desenvolvimento Científico e Tecnológico (CNPq)}
\newcommand{\FAPERJ}{Fundação de Amparo à Pesquisa do Rio de Janeiro (FAPERJ)}
\newcommand{\UFF}{Universidade Federal Fluminense (UFF)}
\newcommand{\PUCrio}{Pontifícia Universidade Católica do Rio de Janeiro (PUC-Rio)}

%--- Defining commands for general information
\newcommand{\Title}{Curriculum Vit\ae}
\newcommand{\Firstname}{Andr\'{e} Luis}
\newcommand{\Middlename}{Albuquerque}
\newcommand{\Lastname}{Reis}
\newcommand{\Initials}{ALA}
\newcommand{\Fullname}{\Firstname\ \Middlename\ \Lastname}
\newcommand{\NameAbbreviation}{André L. A. Reis}
\newcommand{\CiteMe}{\textbf{\Lastname , \Initials}\*} % for citation
\newcommand{\PersonalEmail}{reisandreluis@gmail.com}
\newcommand{\ResearchGroup}{pinga-lab.org/people/andre}
\newcommand{\GitHubProfile}{@andrelreis}
\newcommand{\ORCID}{0000-0002-2225-5106}
\newcommand{\LattesID}{lattes.cnpq.br/1075610796165589}

%--- Defining commands for citations
\newcommand{\Vand}{Oliveira Jr, VC}
\newcommand{\Elder}{Yokoyama, E}
\newcommand{\AC}{Bruno, AC}
\newcommand{\Joao}{Pereira, JMB}
\newcommand{\Jeff}{Araujo, JFDF}
\newcommand{\Takahashi}{Takahashi, D}
\newcommand{\Val}{Barbosa, VCF}
\newcommand{\Medina}{Medina, CD}
\newcommand{\Fredy}{Osorio, FG}
\newcommand{\Cleanio}{Luz-Lima, C}
\newcommand{\Falco}{De Falco, A}
\newcommand{\Caique}{Lima, CDA}
\newcommand{\JoaoFelipe}{Silva, JFC}
\newcommand{\Lanna}{Sinimbu, LIM}
\newcommand{\Frederico}{Gutierrez, FV}
\newcommand{\Walmir}{Pottker, WE}
\newcommand{\LaPorta}{La Porta, FA}
\newcommand{\Mendoza}{Mendoza, LAF}
\newcommand{\Tahir}{Tahir}
\newcommand{\Tommaso}{Del Rosso, T}
\newcommand{\Jesana}{Loreto, JM}
\newcommand{\Renan}{Loreto, RP}
\newcommand{\Cilene}{Labre, C}
\newcommand{\Chaves}{Chaves, JF}
\newcommand{\Merino}{Merino, ILC}
\newcommand{\Saitovitch}{Saitovitch-Baggio, E}
\newcommand{\Guillermo}{Solorzano, G}
\newcommand{\Geronimo}{Perez, G}
\newcommand{\Eloi}{Junior, EBM}
\newcommand{\Amanda}{Santo, AF}
\newcommand{\Angela}{Correa, AAP}
\newcommand{\Pacheco}{Pacheco, MAC}
\newcommand{\Giancarlo}{Brito, GE}
\newcommand{\AraujoWWR}{Araujo, WWR}



%-------------------------
% Template configuration
%-------------------------

%--- Setting main font
\renewcommand{\familydefault}{\sfdefault}

%--- Geometry of the CV
\geometry{%
  margin=12.0mm,
  headsep=0mm,
  headheight=0mm,
  footskip=5mm,
  includehead=true,
  includefoot=true
}

%--- Custom colors
\definecolor{mediumgray}{gray}{0.5}
\definecolor{lightgray}{gray}{0.9}
\definecolor{mediumblue}{HTML}{4233ff}
\definecolor{lightblue}{HTML}{33ffe9}
\definecolor{brown}{HTML}{6e2c00}
\definecolor{lightbrown}{HTML}{eb984e}

%--- No indentation
\setlength\parindent{0cm}

%--- Increase the line spacing
\renewcommand{\baselinestretch}{1.1}
%--- and the spacing between rows in tables
\renewcommand{\arraystretch}{1.25}

%--- Remove space between items in itemize and enumerate
\setlist{nosep}

%--- Customizing the style of the sections
\titleformat{\section} % command
   {\normalfont\Large\mdseries} % format
  {} % label
  {1pt} % sep
  {} % before-code
  [\titlerule] % after-code
\titlespacing*{\section}
  {0pt} % left pad
  {0.1cm} % before
  {0cm} % after
  
%--- Hyperlinks configuration
\hypersetup{
           colorlinks=true,
           linkcolor=mediumblue,
           allcolors=mediumblue,
           pdftitle={\NameAbbreviation{} CV},
           pdfauthor={\NameAbbreviation},
}

%--- Footer Style
\newcommand{\Separator}{\hspace{3pt}|\hspace{3pt}}
\newcommand{\FooterFont}{\footnotesize\color{mediumgray}}
\pagestyle{fancy}
\fancyhf{}
\lfoot{%
  \FooterFont{}
  \NameAbbreviation{}
  \Separator{}
  \Title{}
}
\rfoot{%
  \FooterFont{}
  Last updated: \monthyear\today{}
  \Separator{}
  \thepage\space of \space \pageref*{LastPage}
}
\renewcommand{\headrulewidth}{0pt}
\renewcommand{\footrulewidth}{1pt}
\preto{\footrule}{\color{mediumgray}}

\newdateformat{monthyear}{\monthname[\THEMONTH], \THEYEAR}

%--- Defining new entries for CV in a tabular form

\newcommand{\pad}{\vspace{0.2cm}}
\NewEnviron{entries}{
    \pad
    \begin{tabularx}{\textwidth}{p{0.12\textwidth}p{0.82\textwidth}}
      \BODY{}
    \pad
    \end{tabularx}
}

%--- General macros

\newcommand{\Duration}[2]{\fontsize{10pt}{0}\selectfont \textsf{#1 - #2}}
\newcommand{\Website}[1]{\href{https://#1}{#1}}
\newcommand{\DOI}[1]{DOI: \href{https://doi.org/#1}{#1}}
\newcommand{\GitHub}[1]{GitHub: \href{https://github.com/#1}{#1}}

%--- New command multiline text

\newcommand{\multiline}[3]{{#1} & {\textbf{#2} \newline {#3}}}

%--- Other commands for entries

%-----{\paper}{Year}{{<title>},{<authorship>},{<journal>},{DOI}}
\newcommand{\paper}[5]{%
    {#1} & 
    {\textbf{#2} \newline 
    {#3} \newline 
    \textit{#4}, \DOI{#5}}}
    
%-----{talk}{Year}{{<title>},{<conference>},{<type>},{DOI}}
\newcommand{\talk}[5]{%
                      {#1} & 
                      {\textbf{#2} \newline 
                      {#3}, \textit{#4}, \DOI{#5}}}

%%%%%%%%%%%%%% End Template configurations

%-------------------------------
% The beggining of the document
%-------------------------------
\begin{document}

%------ Begin Header
\begin{minipage}[t]{0.5\textwidth}
  {\fontsize{20pt}{0}\selectfont\Fullname}
\end{minipage}
\begin{minipage}[t]{0.5\textwidth}
  \begin{flushright}
    \Title{}
  \end{flushright}
\end{minipage}
\\[-0.1cm]
\textcolor{black}{\rule{\textwidth}{2pt}}
\begin{minipage}[t]{0.5\textwidth}
   \footnotesize \parbox{0.04\textwidth}{\faEnvelope} Email: \href{mailto:\PersonalEmail}{\PersonalEmail}
  \\
  \footnotesize \parbox{0.04\textwidth}{\aiOrcid} ORCID: \href{https://orcid.org/\ORCID}{\ORCID}
  \\
  \footnotesize \parbox{0.04\textwidth}{\faGithub} GitHub: \href{https://github.com/andrelreis}{\GitHubProfile}
  \\
  \footnotesize \parbox{0.04\textwidth}{\aiLattes} LattesID: \href{http://lattes.cnpq.br/1075610796165589}{\LattesID}
  \\
  \footnotesize \parbox{0.04\textwidth}{\faUsers} Research Group: \href{https://www.\ResearchGroup}{\ResearchGroup}
\end{minipage}
\begin{minipage}[t]{0.5\textwidth}
  \begin{flushright}
  \DGAP
  \\
  \FGEL
  \\
  \UERJ
  \\
   Rua São Francisco Xavier, 524. Rio de Janeiro - RJ. Brazil
  \end{flushright}
\end{minipage}
%------- End Header

\vspace{0.3cm}

%------- CV's Body
\section{Brief Description}
\vspace{0.3cm}
I am currently an Adjunct Professor in the Department of Applied Geology at the Geology School of Rio de Janeiro State University. I am also leading an Exploration Geophysics Laboratory. My research focuses on developing methods for processing and interpreting potential fields. I am experienced in applying Scanning Magnetic Microscopy to characterize magnetic materials and rock samples, proposing new technologies in Paleomagnetism and Rock magnetism. I have interests in Computational and Theoretical Geophysics.

\section{Education}

\begin{entries}
    \Duration{2016}{2020} & \textbf{PhD in Geophysics}, \ON, Brazil 
    \\
    \Duration{2014}{2016} & \textbf{MSc in Geophysics}, \ON, Brazil 
    \\
    \Duration{2007}{2012} & \textbf{BSc in Physics}, \UERJ, Brazil 
\end{entries}

\section{Work}

\begin{entries}
    \Duration{2021}{on} & \textbf{Adjunct Professor}, \UERJ, Brazil
    \\
    \Duration{2020}{2021} & \textbf{Postdoctoral Researcher}, \ON, Brazil
    \\
    \Duration{2013}{2014} & \textbf{Technician}, \UENF, Brazil    
\end{entries}

\section{Awards and Scholarships}

\begin{entries}
    \multiline{\Duration{2020}{2021}}
    {Postdoctoral Fellowship}
    {\CNPq \newline \ON, Brazil}
    \\
    \multiline{\Duration{2016}{2020}}
    {PhD Scholarship}
    {\CNPq \newline \ON, Brazil}
    \\
    \multiline{\Duration{2014}{2016}}
    {MSc Scholarship}
    {\CNPq \newline \ON, Brazil}
    \\
    \multiline{\Duration{2007}{2009}}
    {Scientific Initiation Scholarship}
    {\CNPq \newline \CBPF, Brazil}
\end{entries}

\section{Educational Resources}

\begin{entries}
    \Duration{2021}{on} & \textbf{Geofísica I},\GitHub{andrelreis/geofisica1}
    \\
    \Duration{2021}{on} & \textbf{Geofísica II}, \GitHub{andrelreis/geofisica2}
    \\
    \Duration{2023}{on} & \textbf{Introdução ao Processamento Sísmico}, \GitHub{andrelreis/processamento-sismico}
    \\
    \Duration{2021}{on} & \textbf{Inversão de Dados Geofísicos}, \GitHub{andrelreis/introducao-inversao}
    \\
    \Duration{2021}{2021} & \textbf{Métodos Potenciais}, \GitHub{andrelreis/metodos-potenciais}
\end{entries}

\section{Projects and Grants}

\begin{entries}
    \multiline{2024}
    {Métodos computacionalmente eficientes para a descrição magnética de amostras de rocha em escala micrométrica}
    {\FAPERJ}
\end{entries}

\section{Conference Participations}

\begin{entries}
    \multiline{2019}
    {Equivalent layer technique for estimating magnetization direction}
    {SEG Annual Meeting, San Antonio, TX - USA}
    \\
    \multiline{2017}
    {SED for optimal acquisition design and sensor-to-sample distance applied to scanning magnetic microscopy}
    {Bi-annual meeting of the Latinmag, Querétaro - México}
    \\
    \multiline{2016}
    {Impact of the sensor area, acquisition design and position noise on the estimation of the magnetization distribution within a rectangular rock sample}
    {AGU Fall Meeting, San Francisco, CA - USA}    
\end{entries}

\section{Presentations}

\begin{entries}
    \talk{2023}
    {Teoria do Potencial Aplicada: uma contribuição para a descrição de rochas ígneas em bacias sedimentares}
    {X SAGEO-UERJ}
    {Invited Speaker}
    {10.6084/m9.figshare.24156039.v1}
    \\
    \talk{2020}
    {Inversão de dados magnéticos para estimar as três componentes do campo}
    {Jornada PCI-ON}
    {Invited Speaker}
    {10.6084/m9.figshare.13256657.v1}
    \\
    \talk{2019}
    {Equivalent layer technique for estimating magnetization direction}
    {SEG Annual Meeting}
    {Oral presentation}
    {10.6084/m9.figshare.9899321.v1}
    \\
    \talk{2017}
    {SED for optimal acquisition design and sensor-to-sample distance applied to Scanning Magnetic Microscopy}
    {Bi-annual Meeting of the Latinmag}
    {Oral presentation}
    {10.6084/m9.figshare.9899282.v1}
    \\
    \talk{2016}
    {Impact of the sensor area, acquisition design and position noise on the estimation of the magnetization distribution within a rectangular rock sample}
    {AGU Fall Meeting}
    {Poster presentation}
    {10.6084/m9.figshare.9899213.v1}
\end{entries}

\section{Peer-reviewed Published Papers}

\begin{entries}
    \paper{2023}
    {Computational aspects of the equivalent-layer technique: review}
    {\Vand; \Takahashi; \CiteMe; \Val}
    {Frontiers in Earth Sciences}
    {10.3389/feart.2023.1253148}
    \\
    \paper{2023}
    {Construction of a Hall effect scanning magnetic microscope using permanent magnets for characterization of rock samples}
    {\Jeff; \CiteMe; \Elder; \Medina; \Fredy; \Cleanio; \Falco; \Caique; \JoaoFelipe; \Lanna; \Frederico; \Walmir; \LaPorta; \Mendoza; \Tahir; \Tommaso; \AC }
    {Journal of Magnetism and Magnetic Materials}
    {10.1016/j.jmmm.2022.170304}
    \\
    \paper{2022}
    {Spinel nanoparticles characterization by inverting scanning magnetic microscope maps}
    {\Jesana; \CiteMe; \Renan; \Cilene; \Chaves; \Caique; \AC; \Cleanio; \Merino; \Saitovitch; \Guillermo; \Jeff}
    {Ceramics International}
    {10.1016/j.ceramint.2022.04.149}
    \\
    \paper{2021}
    {Detecting surface-breaking flaws with a Hall effect gradiometric sensor}
    {\Eloi; \Fredy; \Frederico; \Tommaso; \Tahir; \Mendoza; \Cleanio; \Elder; \CiteMe; \Geronimo; \Jesana; \AC; \Jeff}
    {Measurement}
    {10.1016/j.measurement.2020.108808}
    \\
    \paper{2020}
    {Generalized positivity constraint on magnetic equivalent layers}
    {\CiteMe; \Vand; \Val}
    {Geophysics}
    {10.1190/GEO2019-0706.1}
    \\
    \paper{2019}
    {Characterizing Complex Mineral Structures in Thin Sections of Geological Samples with a Scanning Hall Effect Microscope}
    {\Jeff; \CiteMe; \Vand; \Amanda; \Cleanio; \Elder; \Mendoza; \Joao; \AC}
    {Sensors}
    {10.3390/s19071636}
    \\
    \paper{2019}
    {Scanning Magnetic Microscope Using a Gradiometric Configuration for Characterization of Rock Samples}
    {\Jeff; \CiteMe; \Angela; \Elder; \Vand; \Mendoza; \Pacheco; \Cleanio; \Amanda; \Fredy; \Giancarlo; \AraujoWWR; \Tahir; \AC; \Tommaso}
    {Materials}
    {10.3390/ma12244154}
    \\
    \paper{2016}
    {Estimating the magnetization distribution within rectangular rock samples}
    {\CiteMe; \Vand; \Elder; \AC; \Joao}
    {Geochemistry, Geophysics, Geosystems}
    {10.1002/2016GC006329}   
\end{entries}

\section{Defense Committee Participations}

\begin{entries}
    \multiline{2023}
    {Lanna Isabely Morais Sinimbu}
    {MSc defense \newline \PUCrio}
    \\
    \multiline{2023}
    {Victor Lebre Fiaux Rodrigues}
    {MSc defense \newline \UERJ}
    \\
    \multiline{2023}
    {Leonardo Campos João}
    {MSc defense \newline \UERJ}
    \\
    \multiline{2023}
    {Renato Mota Xavier de Meneses}
    {PhD defense \newline \UFF}
    \\
    \multiline{2023}
    {Victor Lebre Fiaux Rodrigues}
    {MSc qualifying \newline \UERJ}
    \\
    \multiline{2022}
    {Rômulo Rodrigues de Oliveira}
    {MSc defense \newline \UFF}
    \\
    \multiline{2022}
    {Guilherme Zequini Gomes}
    {MSc defense \newline \UERJ}
    \\
    \multiline{2021}
    {Bruno Lima de Freitas}
    {Undergraduate thesis defense \newline \UFF}
    \\
    \multiline{2020}
    {Allan Soares Ramalho}
    {Undergraduate thesis defense \newline \UFF}
    \\
    \multiline{2020}
    {Shayane Paes Gonzalez}
    {PhD qualifying \newline \UFF}
\end{entries}

\section{Technical Skills}

\begin{entries}
    \textbf{Programming} & Python, FORTRAN
    \\
    \textbf{Markup} & Markdown, LaTeX
    \\
    \textbf{Graphics} & InkScape, GIMP
    \\
\end{entries}

\section{Languages}

\begin{entries}
    \textbf{Portuguese} & Native
    \\
    \textbf{English} & Advanced
    \\
    \textbf{Italian} & Begginer
    \\
    \textbf{Spanish} & Begginer
\end{entries}
























\end{document}